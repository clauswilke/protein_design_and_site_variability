\documentclass[12pt]{article}
\usepackage{graphicx, subfigure, float, amsmath, amssymb, color}
\usepackage[margin = 1.0 in]{geometry}
%\usepackage[style = nature]{biblatex}

% definition of \customlabel, which is used to label supplementary figures and tables
\makeatletter
\newcommand{\customlabel}[2]{%
\protected@write \@auxout {}{\string \newlabel {#1}{{#2}{}}}}
\makeatother



\title{Amino-acid site variability among natural and designed proteins}
\author{Eleisha L.\ Jackson$^1$, Noah Ollikainen$^2$, Arthur W.\ Covert III$^1$,\\ Tanja Kortemme$^{2,3}$, and Claus O.\ Wilke$^1$}
\begin{document}

\date{\today}
\maketitle

\noindent
$^1$ Institute of Cellular and Molecular Biology, Center for Computational Biology and Bioinformatics, and Department of Integrative Biology, The University of Texas at Austin, Austin, Texas, USA\\
$^2$ Graduate Program in Bioinformatics,University of California San Francisco, San Francisco, California, USA\\
$^3$ California Institute for Quantitative Biosciences (QB3) and Department of Bioengineering and Therapeutic Science, University of California San Francisco, San Francisco, California, USA

\bigskip

\noindent
{\color{red}Text in red: Eleisha, please work on this.}

\noindent
{\color{blue}Text in blue: Material Claus hasn't touched yet.}

%Use \textbf{} to bold tetxt
\begin{abstract}
{\color{blue}
Protein structure prediction software attempts to create protein structures that are structurally similar to natural proteins. We examine how accurately Rosetta, a protein prediction software, reconstructs observed patterns of variability found in natural proteins. We use Rosetta to design proteins and then compare these designed proteins with natural proteins. Our comparisons include site variability, observed distributions at sites and the effects of protein structure on site variability. Proteins designed with a fixed backbone underestimate the amount of site variability observed in natural proteins while proteins designed with a flexible backbone result in more site variability. Intermediate flexibility during design results in site variability patterns that most accurately resemble those found in natural proteins. From these results we conclude that intermediate backbone flexibility during design results in more accurate protein design and that scoring functions that determine acceptable substitutions must improve to account for structural constraints on site variability patterns.
}
\end{abstract}

\section{Introduction}
\label{Introduction}

{\color{blue}
There are many selective pressures that affect the rate at which protein sequences change over time. Some of these important determinants include protein dispensability, expression and protein structure. A protein's structure determines how it can function and interact with other proteins.  Therefore a thorough knowledge of the constraints of protein structure on protein evolution is necessary to understand how proteins function. Many proteins need stable native structure to preserve their function. Understanding how proteins function and evolve is of critical importance for the development of proteins with novel functions, advances in drug therapy and increasing our knowledge of disease.  
}

{\color{blue}
It is well documented that protein structure has an influence on variability seen at sites \cite{Franzosa2009, Ramsey2011}. Computational protein design can be used develop and analyze protein structures allowing us to further our knowledge of the effects of protein structure on sequence evolution. In fact recent work using knowledge gained from computational design has resulted in the design of proteins that bind to an influenza virus \cite{Fleishman2011}. During design, protein sequences are optimized for stability \cite{Butterfoss2006, Das2008}. Protein stability has been shown to be an important selective force on proteins \cite{Drummond2008}. Therefore these optimized structures can be used to assess the constraints of structural stability on protein sequences. Comparing designed and natural proteins will allow us to understand how protein structure, and in particular, protein stability shape observed sequence patterns.
}


Here, we carried out a systematic comparison between alignments of natural sequences and the corresponding alignments of designed sequences, for several different design conditions. We considered two distinct data sets, one of whole protein structures and one of individual protein domains. We analyzed which design conditions produced sequence alignments that were most similar to natural sequence alignments. We also analyzed by which parameters designed proteins differed the most from natural sequences. Overall, we found that proteins designed with a flexible backbone and using an intermediate amount of backbone flexibility were the most similar to natural proteins. However, substantial differences between designed and natural proteins remained even under the most advantageous design conditions. In particular, designed proteins tended to have too many polar and too few hydrophobic residues in the core, and they also tended to have cores that were too variable and/or surfaces that were too conserved. These trends were exacerbated for longer proteins.


\section{Methods}
\label{Methods}

\subsection{Data sets}

We analyzed two data sets, one of whole yeast proteins and one of protein domains. The yeast-proteins data set was taken from Ref.\ \cite{Ramsey2011} and comprised 38 protein structures homologous to an open reading frame in \emph{Saccharomyces cerevisiae}. For each of those structures, we had at least 50 homologous natural sequences, also taken from Ref.\ \cite{Ramsey2011}. The protein-domain data set was taken from Ref.\ \cite{OllikainenKortemme} and comprised 40 protein domains. Only domains with at least one crystal structure in the Protein Database (PDB) and at least 500 sequences in the Pfam Database were selected for this data set. Domains were selected in order to represent several different types of protein folds and omains were also restricted to a length less than or equal to 150 amino acids. For each of these protein domains, we obtained alignments of homologous natural sequences from the Pfam database \cite{Pfam}, as described \cite{OllikainenKortemme}.  

\subsection{Protein design}

For each structure in both data sets, we computationally designed 500 variants each, using multiple design methods. All design methods we used are implemented in the protein-design software Rosetta \cite{generic-rosetta-reference}. First, we used standard fixed-backbone design \cite{fixed-design}. In this method, the protein backbone remains fixed and only amino-acid side chains are allowed to move. Second, we used the flexible-backbone method Backrub \cite{Smith2008}, which first generates an ensemble of alternative backbones and then designs side chains onto these backbones. The Backrub method takes as input a temperature parameter that determines the extent of backbone movements that occur during design. A temperature of zero corresponds to the fixed-backbone case while a temperature in excess of 1 allows substantial backbone movements. Here, we used temperatures spanning from 0.03 to 2.4. For the protein-domain data set, we also carried out one additional design method, called ``Soft''. This method which is similar to the fixed backbone method but the energy function used during sequence design dampens the weight of the repulsive Lennard-Jones (LJ) potential term \cite{OllikainenKortemme}.  Protein designs for the protein-domain data set have been previously published \cite{OllikainenKortemme}, while the designs for the yeast-proteins data set were newly generated for the present study.

\subsection{Data analysis}

We quantified the variability of sites in amino-acid alignments using site entropy $H_i$, defined as $H_i=\sum_{j}p_{ij}\ln p_{ij}$. Here, $p_{ij}$ is frequency of amino acid $j$ in alignment column $i$, and the sum runs over all amino acids. We compared amino-acid distributions of designed sequences to those of natural sequences using the Kullback-Leibler (KL) divergence. The KL divergence $D^\text{KL}_i$ is defined as $D^\text{KL}_i= \sum_j  p_{ij} \ln  (p_{ij}/q_{ij})$, where $q_{ij}$ is the frequency of amino acid $j$ in column $i$ of the reference alignment, and $p_{ij}$ is the corresponding frequency in the alignment that is being compared to the reference alignment. The sum runs over all amino acids.  When calculating frequencies used for the KL divergence we corrected for the presence of frequencies of zero by adding  $\frac{1}{20}$ to each amino acid count before calculating the frequencies. The KL divergence is inherently an asymmetric distance measure, comparing a probability distribution of interest to a reference distribution. Unless noted otherwise, we always used natural sequence alignments to calculate the reference frequencies $q_{ij}$ and designed sequence alignments to calculate the frequencies $p_{ij}$. Throughout this work, we calculated $D^\text{KL}_i$ separately at each site $i$ in a protein, and then averaged the $D^\text{KL}_i$ values for all sites in a protein to obtain a mean KL divergence for that protein. 

We calculated Relative Solvent Accessibility (RSA) of residues by first calculating the absolute Solvent Accessibility (SA) for each residue, using the software DSSP \cite{Kabsch1983}. For each protein, we extracted the chain of interest from the PDB structure and ran DSSP only on that chain. We calculated RSA by dividing the SA value for each residue by the maximum possible SA value, as given in Ref.\ \cite{Tien}. 

\section{Results}
\label{Results}


We wanted to assess the extent to which the sequence space of computationally designed proteins overlaps with the sequence space occupied by homologous natural proteins. Our general approach was to compare alignments of designed protein sequences to alignments of homologous natural sequences, for approximately 80 distinct protein structures. For each structure, we considered several different design methods (see Methods for details), and we designed 500 sequences for each structure and method. The protein structures we considered were subdivided into two distinct data sets, a data set of 38 yeast protein structures previously analyzed in Ref.\ \cite{Ramsey2011} and a data set of 40 protein domains previously analyzed in  Ref.\ \cite{OllikainenKortemme}. Throughout this study, we analyzed these two data sets separately, because they corresponded to structures of substantially different sizes. The mean number of amino acids per structure was 215.4 in the yeast-proteins data set and 86.1 in the protein-domains data set.

\subsection{Overall site variability}
\label{SiteVariability}

We first compared overall amino-acid variability in designed and natural proteins. We assessed amino-acid variability at individual sites by calculating the entropy $H_i$ at each site $i$ in alignments of either designed or natural proteins. We then calculated the mean entropy over all sites in each alignment and used that quantity as a measure of the overall amino-acid variability in the alignment.

We found that protein design using a fixed backbone generally yielded insufficient site variability compared to natural sequences (Fig.~\ref{MeanEntropyComparison}).  This results was magnified in the smaller, protein domains. In fact, for the protein domains ,the most variable proteins under fixed-backbone design showed only about as much variability as the least variable natural proteins. Overall, there was a significant shift towards higher variability in natural proteins relative to proteins designed with fixed backbone (paired $t$ test, $P = 1.447 $ $\times$  $10^{-10}$ for the yeast-proteins data set and $P<10^{-15}$ for the protein-domain data set). When switching from fixed-backbone design to variable-backbone design, we found that overall site variability increased. Further, site variability increased monotonously with the degree of backbone flexibility allowed during design, as measured by the design temperature (Fig.~\ref{MeanEntropyComparison}). At the highest temperatures, site variability in designed proteins consistently exceeded that of natural proteins. 

Proteins designed at intermediate temperatures of had site variability that mostly closely resembled that of natural proteins (paired $t$ test, $P= 0.0006$ for T = 0.03 and $P= 02.22$ $\times$  $10^{-7}$  for T  = 0.1 for the yeast-proteins data set; $P= 0.353$ for T = 0.9 and $P = 0.012 $ for T = 1.2 for the protein-domain data set). In addition, this range of temperatures was lower for the yeast proteins ( T = 0.03 - 0.1 versus T = 0.9 -  1.2 for the protein-domain data set).  However, at those temperatures, natural proteins generally showed a larger spread in variabilities than designed proteins did (Brown–Forsythe test for equal variances,  $P= 0.0003$ for T = 0.03 and $P=0.0002$ for T = 0.1 for the yeast-proteins data set and $P= 7.29$  $\times$  $10^{-6}$ for T = 0.9 and  $P= 2.34$ $\times $ $10^{-5}$ for T = 1.2  for the protein-domain data set).

\subsection{Amino-acid distributions}
\label{AminoAcidDistributions}

We next compared amino-acid distributions between designed and natural sequences. First we looked at overall amino acid frequencies. We found that by-and-large, amino acid frequencies in designed proteins mirrored those in natural proteins (Figs.~\ref{AAFreqsYeastProteins} and \ref{AAFreqsProteinDomains}). The biggest differences arose in Cys, Pro, His, Trp, Phe, and Ala. Overall, we observed that hydrophobic residues tended to be under-represented in designed proteins whereas hydrophilic residues tended to be over-represented. This trend was stronger in the protein core than on the surface (Figs.~\ref{AAFreqsYeastProteins} and \ref{AAFreqsProteinDomains}). We also observed that the longer proteins in the yeast-proteins data set showed larger deviations between designed and natural sequences than the shorter proteins in the protein-domains data set. Finally, when comparing different design methods and design temperatures, we found that differences in amino-acid distributions were relatively minor (not shown).

Even if overall amino-acid distributions are approximately correct, the amino-acid distributions at individual sites can be poorly predicted \cite{Ramsey2011}. Therefore, we next compared, separately at each site, the similarity between amino-acid distributions in natural proteins and those in designed proteins. To carry out this comparison, we employed the Kullback-Leibler (KL) divergence {\color{red}ref?}, which measures how similar one probability distribution is to a reference distribution. A KL divergence of zero implies that the distributions are identical. The higher the KL divergence, the more dissimilar the focal distribution is to the reference distribution. (Note that KL divergence is not symmetric: if we swap the focal and the reference distribution, we will generally obtain a different KL divergence value.) We calculated the KL divergence at each site in each protein, and then averaged over sites within a protein to obtain a mean similarity score for each protein. As a control, we also randomly split the alignment of natural sequences for each protein structure into two halves and calculated the mean KL divergence of natural sequences against themselves.

First, in all comparisons, we found that the KL divergence of designed relative to natural sequences was much bigger than the KL divergence of natural sequences relative to themselves (Figs.~\ref{AADisFig1} and~\ref{NoahAADisFig1}). This finding indicates a substantial discrepancy between designed and natural sequences at individual sites. Second, we found that the mean KL divergence decreased with increasing design temperature (Figs.~\ref{AADisFig1}A and~\ref{NoahAADisFig1}A). Thus, according to the KL divergence measure, structures designed with the most flexible backbones had the most similar amino-acid distributions to those found in natural sequences.

However, the result that sequences designed at the highest temperatures are the most similar to natural sequences may be an artifact of the KL divergence measure. As design temperature increases, amino-acid variability increases, and amino-acid distributions become more uniform. A more uniform distribution is generally going to display more overlap with any given distribution than a more localized distribution, if the localized distribution is not correct. Thus, the decrease in KL divergence with increasing temperature may simply reflect the broadening of the distribution, not an actual improvement in reproducing natural amino-acid distributions. To assess whether amino-acid distributions in designed sequences were simply broadening with increasing temperature, or whether they were actually converging on the natural distributions, we carried out a second set of comparisons. We rank-ordered amino acids by frequency at each site in each protein, and then calculated the KL divergence of the rank-ordered distributions. This comparison considers only the shape of the distribution and does not assess whether the correct amino acids are present at individual sites. This second comparison generally found much lower KL divergence levels, even though still not as low as what was found for the control comparison of natural sequences with themselves (Figs.~\ref{AADisFig1}B and~\ref{NoahAADisFig1}B). 

More importantly, now KL divergence reached a minimum around a temperature of 0.3 (yeast proteins, Fig.~\ref{AADisFig1}B) to 1.2 (protein domains, Fig.~\ref{NoahAADisFig1}B) and rose again beyond that value. This finding indicates that higher design temperatures do not unequivocally produce more natural amino-acid distributions. Instead, there is an intermediate temperature, approximately coinciding with the temperature at which overall sequence variability matches best, at which amino acid distributions also are most similar.

\subsection{Site variability and solvent accessibility}
\label{ProteinStructure}

The previous analyses demonstrated that while designed proteins overall look similar to natural proteins, there are also important differences. We next wanted to identify whether these differences were present uniformly throughout the structure or could be located to specific structural regions. In our analysis of amino-acid distributions, we had already seen that amino-acid distributions seemed to deviate more at buried sites than at exposed sites (Figs.~\ref{AAFreqsYeastProteins} and~\ref{AAFreqsProteinDomains}).

We first plotted site variability against relative solvent accessibility (RSA, a dimensionless number from 0 to 1 measuring the relative solvent exposure of individual residues) for individual proteins. See Fig.~\ref{Entropy_vs_RSA_example} for one example. We generally found that site variability displayed a substantial spread even for sites of very similar RSA. At the same time, there was an overall trend for sites with higher RSA to be more variable than sites with lower RSA. This trend was generally stronger in flexible backbone designs than in fixed backbone designs (Fig.~\ref{Entropy_vs_RSA_example}).

To analyze the relationship between site variability and RSA more systematically, we calculated the correlation between these two quantities for all proteins (Figs.~\ref{Correlation_figure} and~\ref{Correlation_figure_Noah}). On average, natural sequence alignments showed a higher correlation than alignments of designed sequences, regardless of design method. 

Intermediate design temperatures showed the highest correlations, but correlations were nevertheless significantly lower in designed proteins than in natural proteins (paired $t$ test, $P=  3.78$  $\times$  $10^{-10}$ [$T=0.3$, yeast proteins] and $P= 2.10$ $\times$  $10^{-5} $ [$T=0.3$, protein domains]).  We also investigated whether the designed proteins with the highest correlations corresponded to the natural proteins with the highest correlations, and found this generally to be the case (Figs.~\ref{Correlation_figure}B and~\ref{Correlation_figure_Noah}B).

Our finding that correlations between site entropy and RSA are lower in designed proteins than in natural proteins indicates that, in designed proteins, site variability is too uniform across different solvent exposure states. In short, designed proteins are either too variable in the core or too conserved on the surface. To obtain a clearer picture of how exactly designed proteins differed from natural proteins, we once more considered the distributions of mean site entropies, but now calculated separately for buried sites ($\text{RSA}\leq0.05$), for partially buried sites ($0.05<\text{RSA}\leq0.25$), and for exposed sites ($\text{RSA}>0.25$). Figure~\ref{Mean_Entropy_Surface_Core} shows the medians of these distributions. For designed proteins, the mean site variabilities of exposed and of partially buried sites are close in magnitude while the mean site variabilities of buried sites are generally consistently lower. By contrast, in natural sequences exposed sites show much more variability than partially buried sites.

If buried sites are too variable or exposed sites too conserved in designed proteins, we reasoned that hybrid designs, in which buried sites were taken from sequences designed at a lower temperature and exposed sites from sequences designed at a higher temperature, should display correlations more similar to those seen in natural proteins. 

According to Fig.~\ref{Mean_Entropy_Surface_Core}, for the yeast proteins buried and partially buried sites in designed proteins had site variability most similar to that of natural sequences in proteins designed with a fixed backbone or in proteins design temperature of $T=0.03$. In the protein domains these were sites within designed proteins with a temperature of $T=0.3$ to $T = 0.6$  whereas exposed sites in designed proteins had site variability most similar to that of natural sequences at a design temperature of $T= 0.1$ (yeast proteins) and $T  = 1.2$ (protein domains).  We thus built our hybrid designs by combining sites from these temperatures. We found that indeed, the site-entropy--RSA correlation in hybrid designs was nearly identical to that in natural sequences (Fig.~\ref{Mixed_RSA_Entropy}). {\color{blue}\emph{I want to see the revised results before I can finish this paragraph.}}

\section{Discussion}

We have compared site variability and amino-acid distributions in designed and natural proteins, for two distinct data sets. One data set consisted of 38 yeast proteins, and the other consisted of 40 protein domains. Structures in the yeast-proteins data set were, on average, much larger than structures in the protein-domain data set, while alignments of natural sequences in the protein-domain data set were somewhat more variable than those in the yeast-proteins data set. We have found that proteins designed with a flexible backbone, using an intermediate design temperature were overall the most similar to natural proteins. Overall amino-acid frequencies in designed proteins were similar, though not identical, to those in natural proteins. However, amino-acid frequencies at individual sites showed substantial deviations. Finally, we have found that site variabilities in designed proteins is too uniform across different solvent exposure states of residues. Designed proteins have either cores that are too variable or surfaces that are too conserved.



In the past, native sequence recovery has been used to assess design accuracy (find citation). However,  native sequence recovery may not always been a good indicator of design accuracy especially when examining different sequences that are compatible for a specific structure (Cite two works of when it was used). One goal of design might be to find sequences that fold to a specific structure. In order to do this one would design a series of structural ensembles that are similar to native structure and then find low energy sequences for each of these designed ensembles. Sequence recovery is not a good indicator of design accuracy in these cases. Methods where you compare difference properties of the designed structures to the a set of reference structure are provide an alternate method of assessing design accuracy. One previous study includes looking at sequence profile similarity, sequence entropy, and amino acid co-variation between designed and natural proteins using protein domains \cite{OllikainenKortemme}. 



In this study they looked at profile similarity, sequence entropy, and amino acid co-variation at sites within designed and natural sequences.  Sequence entropy for each site was calculated by Shannon Entropy (insert citation) for sites. They used profile similarity to account for similarities in amino acid distributions as opposed to our measurement of using KL Divergence to quantity amino acid distributions. In regards to structural analysis they used half of the dataset, examine structural variation (determined by RMSD) (insert citation).  The same set of protein domains was used in our analysis.  In both analyses it was found that intermediate flexibility resulted in more similar patterns as compared to natural proteins and found similar trends in site variability. 
In our work, we explicitly examine the relationship between structure and sequence variability as measured by the correlation between RSA and site entropy and compare natural and designed alignments. This allows us to examine the differences between buried and exposed residues.
By creating hybrid designs, we better understand how different amounts of backbone flexibility better recapitulate observed patterns within natural proteins. We also directly compared amino acid frequencies for each of the 20 canonical amino acids within natural and designed proteins.



In our analysis we used two distinct datasets. One was a set containing 40 protein domains. These protein domains were constrained to be less than 150 amino acids in length and had a mean length of 86.1.(structured regions?) Our second dataset was comprised of 38 whole yeast proteins with a mean length of 215.4. This size difference may account for the differences in optimal design temperatures. Large cores may lead to larger conserved regions whose site variability patterns are better recaptured at lower designed temperatures. Due a desire to represent a large variety of different protein folds, the protein-domain data set was more more diverse than the yeast protein data set. 


Designed protein sequences are optimized for stability while natural proteins experience selective pressure for stability. Therefore we expect there to be overlapping similarities between the two in regards to observed sequence properties. However, we do no expect there to be perfect agreement between the site variability patterns observed in nature and those seen in the designed proteins.  Natural proteins experience a number of other selective pressures (ex.  selection for translational accuracy, protein dispensability, functional interactions). These selective pressures also play a role in shaping the observed sequence properties of natural proteins. 


We have found that in designed proteins polar residues were over-represented and hydrophobic amino acids are over-represented within the core. These discrepancies suggest a need for improvement within the design algorithm. Rosetta uses a scoring function that analyzes the energy of a sequence for a target structure (\cite{DasBaker}]One method might be to refine the parameters that are used in energy terms of the scoring function. I suggest improvements to the method that Rosetta uses to parameterize the energy function that affects the number of hydrophobic amino acids on the surface. Rosetta uses reference energies for each amino acid to control for amino acid composition within the designed sequences (\cite{Jacak2012}). These energies favor more polar protein surfaces and control for surface hydrophobicity. Improvements to these parameters or modifications to the way scoring function handles hydrophobicity on the surface or the core might further improve the protein design algorithm.  There could also be improvements in the design algorithm that better address core packing. Methods such as RosettaHoles (\cite{ShefflerBaker})that allow for assessment of protein core packing can been used in future algorithm refinement. 

In our analysis of approximately 80 protein structures ,we found that proteins designed with an intermediate amount of backbone flexibility exhibit site variability patterns mostly closely resembling that of natural proteins.  However, the optimal range for designed was different depending on protein size. We also found differences within the designed proteins when comparing the amino acid distributions at sites as compared to natural proteins. This was quantified by the large KL divergence values obtained for the designed proteins. Intermediate levels of backbone flexibility result in amino acid distribution patterns most similar to amino acid site distributions observed in natural proteins. In addition, intermediate design temperatures showed the highest correlation between RSA and site variability (as measured by entropy). However, even at the temperature where the design proteins had the higher correlation coefficient (T $\sim$ 0.3 for both data sets), the designed proteins exhibited a lower correlation than that of natural proteins. This implies that for designed proteins, proteins have either surfaces that are too conserved or cores that too variable. In addition, designed proteins had too many polar and too few hydrophobic amino acids within their cores. This trend was exacerbated for larger proteins. In order to better control for amino acid frequencies within the core and surface, the scoring function within Rosetta might be modified to reflect difference within the protein?s core versus its surface. 



\bibliographystyle{plain} %"style
\bibliography{ProjectBib} %expected file "my refs.bib"

\cleardoublepage

\section{Figures}

\begin{figure}[H]
%\centerline{\includegraphics[width = 3in]{figures/Mean_Entropy_vs_Temp_Boxplot.pdf}\includegraphics[width = 3in]{figures/Mean_Entropy_vs_Temp_Boxplot_Noah.pdf}}
\centerline{\includegraphics[width = 6in]{figures/Mean_Entropy_vs_Temp_Combo_Boxplot.pdf}} %\includegraphics[width = 3in]{figures/Mean_Entropy_vs_Temp_Boxplot_Noah.pdf}}
\caption{Mean site entropy for designed and natural proteins. Each boxplot represents the distribution of mean site entropies within the respective dataset (left: yeast proteins; right: protein domains). ``FB'' refers to fixed-backbone design. Temperature values refer to the design temperature used during the Backrub design method. ``NS'' refers to natural sequences. ``Soft'' refers to the Soft design method. We find generally that more flexible backbones during design allow for more site variability. Intermediate temperatures produce site variabilities most similar to those seen in natural sequences.} Overall, natural sequences in the protein-domains data set are more variable than are those in the yeast-proteins data set.
\label{MeanEntropyComparison}
\end{figure}


\begin{figure}[H]
\centerline{\includegraphics[width = 5in]{figures/Duncan_Freq_Combo_Plots_06.pdf}}
\caption{Amino-acid frequencies in designed and natural proteins. Frequencies were calculated over all sites in all proteins belonging to the yeast-proteins data set. For designed proteins, only flexible-backbone designs with design temperature 0.6 were considered. Top: overall frequencies. Middle: frequencies at exposed sites (defined as sites with $\text{RSA}>0.05$). Bottom: frequencies at buried sites (defined as sites with $\text{RSA}\leq0.05$). {\color{blue}\emph{Should this be a different temperature?} }  }
\label{AAFreqsYeastProteins}
\end{figure}


\begin{figure}[H]
\centerline{\includegraphics[width = 6in]{figures/Mean_KL_vs_Temp_Boxplot.pdf}}
\caption{Mean Kullback-Leibler (KL) divergence for designed and natural proteins, shown for the yeast-proteins data set. A higher KL divergence indicates that the amino-acid distributions at sites in designed proteins are less similar to the corresponding distributions in the natural proteins. ``FB'' refers to fixed backbone design, and ``NS'' refers to the control case where natural sequences are compared to themselves. (A) KL divergence calculated from the relative frequencies of the 20 amino acids. (B) KL divergence calculated from rank-ordered frequency distributions. The most common amino acid in the reference distribution is compared to the most common amino acid in the focal distribution, the same is done for the second-most common amino acid, and so on, irrespective of the type of amino acids.}
\label{AADisFig1}
\end{figure}


\begin{figure}[H]
\centerline{\includegraphics[width = 6in]{figures/Cor_Mean_Entropy_RSA_Combination_Plot.pdf}}
\caption{Distributions of correlation coefficients between site entropy and RSA, for the yeast-proteins data set. ``FB'' indicates fixed-backbone design, and ``NS'' indicates natural sequences. (A) Distributions represented as boxplots. (B) Correlation coefficients for individual proteins. Lines connect identical structures in the different design conditions. The color shading represents the strength of the correlation for the natural sequence alignment. In general, natural proteins display a stronger correlation between site entropy and RSA than designed proteins.}
\label{Correlation_figure}
\end{figure}


\begin{figure}[H]
%\centerline{\includegraphics[width = 3in]{figures/Mean_Entropy_Position_Lineplot.pdf}\includegraphics[width = 3in]{figures/Mean_Entropy_Position_Lineplot_Noah.pdf}}
\centerline{\includegraphics[width = 6in]{figures/Mean_Entropy_Position_Lineplot_Combo.pdf}}
\caption{Median of the distribution of mean sequence entropies for designed and natural sequences, calculated separately for buried, partially buried, and exposed residues (left: yeast proteins; right: protein domains). We defined buried sites as those with $\text{RSA}\leq 0.05$, partially buried as those with $0.05<\text{RSA}\leq0.25$, and exposed as those with $\text{RSA}>0.25$. Connecting lines are meant as a guide to the eye. Note the difference in the $x$-axis scales between the two graphs.}
\label{Mean_Entropy_Surface_Core}
\end{figure}


\begin{figure}[H]
%\centerline{\includegraphics[width = 3in]{figures/Duncan_Mixed_Temp_Correlation_Plot.pdf}\includegraphics[width = 3in]{figures/Noah_Mixed_Temp_Correlation_Plot.pdf}}
\centerline{\includegraphics[width = 6in]{figures/Combo_Mixed_Temp_Correlation_Plot.pdf}}
\caption{Distribution of correlation coefficients between RSA and site entropy for hybrid designs and for natural proteins. {\color{red}For the hybrid designs, buried and partially buried sites were taken from sequences designed at one temperature, and exposed sites were taken from sequences designed at a different temperature. For the hybrid designs, the correlation coefficients were similar to those of natural sequences (paired $t$ test,  $P=0.532$ (T = FB, 0.1)  and  $P= 8.09 $ $\times$ $10^{-8}$ (T = 0.03, 0.1) [yeast proteins], $P= 5.74$  $\times$ $10^{-5}$ (T = 0.3, 1.8) and $P= 0.123$ (T = 0.6, 1.8)  [protein domains])}}
\label{Mixed_RSA_Entropy}
\end{figure}


\cleardoublepage

\section{Supporting Figures}

\centerline{\includegraphics[width = 5in]{figures/Noah_Freq_Combo_Plots_06.pdf}}

\noindent Figure S1. Amino-acid frequencies in designed and natural proteins. Frequencies were calculated over all sites in all proteins belonging to the protein-domains data set. For designed proteins, only flexible-backbone designs with design temperature 0.6 were considered. Top: overall frequencies. Middle: frequencies at exposed sites (defined as sites with $\text{RSA}>0.05$). Bottom: frequencies at buried sites (defined as sites with $\text{RSA}\leq0.05$) {\color{blue}\emph{Once again should this be a different temperature?} } .

\customlabel{AAFreqsProteinDomains}{S1}

\newpage

\centerline{\includegraphics[width = 6in]{figures/Mean_KL_vs_Temp_Boxplot_Noah.pdf}}
\noindent Figure S2. Mean Kullback-Leibler (KL) divergence for designed and natural proteins, shown for the yeast-proteins data set. A higher KL divergence indicates that the amino-acid distributions at sites in designed proteins are less similar to the corresponding distributions in the natural proteins. ``FB'' refers to fixed backbone design, and ``NS'' refers to the control case where natural sequences are compared to themselves. (A) KL divergence calculated from the relative frequencies of the 20 amino acids. (B) KL divergence calculated from rank-ordered frequency distributions. The most common amino aicd in the reference distribution is compared to the most common amino acid in the focal distribution, the same is done for the second-most common amino acid, and so on, irrespective of the type of amino acids.

\customlabel{NoahAADisFig1}{S2}

\newpage


\centerline{\includegraphics[width = 6.5in]{figures/RSA_vs_Entropy_1PV1_Combination_Plot.pdf}}

\noindent Figure S3. Site entropy versus Relative Solvent Accessibility (RSA) for designed and natural sequence alignments of the protein S-formylglutathione hydrolase (PDB: 1PV1, chain A). {\color{blue}\emph{[Methods?$\rightarrow$ ]} RSA values are calculated from the published PDB structure.} Natural sequences exhibit a clear trend of higher site variability at higher RSA values. The flexible backbone designs exhibit a similar trend but the fixed backbone designs do not.
\customlabel{Entropy_vs_RSA_example}{S3}

\newpage

\centerline{\includegraphics[width = 6in]{figures/Cor_Mean_Entropy_RSA_Combination_Plot_Noah.pdf}}
\noindent Figure S4. Distributions of correlation coefficients between site entropy and RSA, for the yeast-proteins data set. ``FB'' indicates fixed-backbone design, ``Soft'' indicates soft backbone design, and ``NS'' indicates natural sequences. (A) Distributions represented as boxplots. (B) Correlation coefficients for individual proteins. Lines connect identical structures in the different design conditions. The color shading represents the strength of the correlation for the natural sequence alignment. In general, natural proteins display a stronger correlation between site entropy and RSA than designed proteins.

\customlabel{Correlation_figure_Noah}{S4}

\newpage

\centerline{\includegraphics[width = 6in]{figures/Combo_Lineplot_Mixed_Temp_Correlation_Plot.pdf}}
\noindent Figure S4.  Distribution of  the correlation coefficients between RSA and site entropy for hybrid designed and natural proteins. Colors correspond to the magnitude of the correlation coefficient in the natural proteins. Hybrids were created by taking sequences for buried residues and partially residues from one temperature and exposed residues from another temperature (left: yeast proteins, right: protein domains). "NS" refers to the natural sequences and "FB" refers to fixed backbone. 
\customlabel{Combo_Mixed_Entropy_Lineplot}{S5}

\newpage


\centerline{\includegraphics[width = 6in]{figures/Combo_Lineplot_Mixed_Temp_Correlation_Lineplot.pdf}}
\noindent Figure S5. Comparison of correlation coefficients between RSA and site entropy hybrid designed and natural proteins. For the hybrid designs, buried and partially buried sites were taken from proteins designed with a fixed backbone [yeast proteins] or a temperature of $T = 0.6$ [protein domains]. Exposed residues were taken from proteins designed with a temperature of $ T =  0.1$ [yeast proteins] or $T = 1.8$ [protein domains] . The line indicates the function, $y(x) = x$. 

\customlabel{Mixed_Entropy_Correlation_Plot}{S5}

\end{document}